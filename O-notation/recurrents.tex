\documentclass[11pt,draft]{article}
\usepackage[utf8]{inputenc}
\usepackage[russian]{babel}
\usepackage[final]{graphicx}
%\usepackage{amssymb}
%\usepackage{amsmath}
%%

\usepackage{latexsym,amssymb,amsthm}
\usepackage{amsfonts}
%\usepackage[T2A]{fontenc}
\usepackage{amsmath}


\usepackage{indentfirst}

 \voffset -24.5mm
 \hoffset -5mm
 \textwidth 173mm
 \textheight 240mm
 \oddsidemargin=0mm \evensidemargin=0mm

\newtheorem{theorem}{Теорема}[section]
\begin{document}
\title{Рекуррентные соотношения}
\maketitle
Есть три стандартных способа решения рекуррентных соотношений:
\begin{enumerate}
  \item Метод подстановки (substitution method);
  \item Метод итераций (iteration method);
  \item Основная теорема о рекуррентных соотношениях (master theorem).
\end{enumerate}

Метод постановки основан на простой идее: угадать ответ и затем доказать его по индукции.
Приведем пример:
\begin{gather*}
  T(n) = 2 T(\lfloor n / 2 \rfloor) + n.
\end{gather*}
Предполагаем, что $T(n) = \mathcal O( n \log n)$.
Пусть эта оценка верна для $\lfloor n / 2 \rfloor$.
Тогда
\begin{equation}
  T(n) \leqslant 2 ( c \lfloor n / 2 \rfloor \log(\lfloor n / 2 \rfloor) + n
  \leqslant cn \log (n / 2) + n = cn \log n - cn \log 2 + n = cn \log n - cn + n 
\end{equation}

Метод итерации заключается в преобразовании рекуррентного соотношения в сумму. Часто первые несколько членов
помогают угадать ответ, который потом можно доказать по индукции (это может оказаться проще, чем расписывать всю сумму целиком <<до конца>>).

Главный интерес для нас сейчас представляет следующая теорема, которая имеет место для соотношений вида
\begin{gather}
  \label{main_recur}
  T(n) = aT(n / b) + f(n),
\end{gather}
где $a \geqslant 1$, $b > 1$ --- некоторые постоянные, а $f$ --- НСНМ положительная функция.
Такое соотношение возникает, если алгоритм разбивает задачу размера $n$ на $a$ подзадач размера $n / b$, при этом совершая $f(n)$ дополнительной работы
(например, объединение результатов).

В качества примера вспомним про {\cal Merge-Sort}: $a = b = 2$, $f(n) = \Theta(n)$.

Можно показать, что об округлении допустимо не заботиться.
\begin{theorem}[Основная теорема о рекуррентных соотношениях]
 Пусть рекурректное соотношение определено формулой  
 \eqref{main_recur}. Тогда:
 \begin{enumerate}
   \item Если  $f(n) = \mathcal O\left(n^{\log_b a-\varepsilon}\right)$ для некоторого $\varepsilon > 0$, то
     $T(n) = \Theta\left(n^{\log_b a}\right)$.
   \item Если $f(n) =  \Theta \left(n^{\log_b a}\right)$, то $T(n) = \Theta\left(n^{\log_b n} \log_2 n \right)$.
   \item Если $f(n) = \Omega \left(n^{\log_b a + \varepsilon} \right)$ для некоторого $\varepsilon > 0$ и если $a f(n / b) \leqslant c f(n)$
     для некоторой константы $c < 1$ и достаточно больших $n$, то $T(n) = \Theta(f(n))$.
 \end{enumerate}
\end{theorem}

\end{document}

